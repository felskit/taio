\documentclass[12pt,a4paper]{article}
\usepackage[T1]{fontenc} 
\usepackage[utf8]{inputenc}
\usepackage{amsmath}
\usepackage{amsthm}
\usepackage{amssymb}
\usepackage{graphicx}
\usepackage{tabularx}
\usepackage{multirow}
\usepackage[hidelinks]{hyperref}
\usepackage[polish]{babel}
\usepackage[useregional=numeric]{datetime2}
\usepackage[a4paper,margin=2cm]{geometry}

\newtheorem{lem}{Lemat}
\newtheorem{thm}{Twierdzenie}
\theoremstyle{definition}
\newtheorem{defn}{Definicja}

\DeclareMathOperator{\ability}{ability}
\DeclareMathOperator{\need}{need}
\DeclareMathOperator{\assign}{assign}
\DeclareMathOperator{\missing}{missing}

\begin{document}

\begin{titlepage}
	\centering
	{\Large Wydział Matematyki i Nauk Informacyjnych\\Politechniki Warszawskiej \par}
	\vspace{1cm}
	\includegraphics[width=0.2\textwidth]{res/img/logo.png} \par
	\vspace{4cm}
	{\LARGE Teoria Algorytmów i Obliczeń\\Laboratorium - Etap 1 \par}
	\vspace{0.5cm}
	{\LARGE Specyfikacja wstępna \par}
	\vspace{2cm}
	{\large Adrian Bednarz,\\Bartłomiej Dach,\\Tymon Felski \par}
	\vspace{2cm}
	{\large Wersja 1.0 \par}
	\vspace{0.5cm}
	{\large \today \par}
\end{titlepage}

\noindent
Lista zmian:\\[.5\baselineskip]
\begin{tabularx}{\textwidth}{|l|l|X|l|}
	\hline
	\textbf{Data} & \textbf{Autor} & \textbf{Opis} & \textbf{Wersja} \\
	\hline
	15.10.2017 & Tymon Felski & Stworzenie szablonu dokumentu & 1.0 \\
	\hline
\end{tabularx}

\newpage
\tableofcontents
\newpage

\section{Opis problemu}

\section{Algorytm}

\section{Dowód poprawności}

W~tym rozdziale wykażemy związek między postawionym problemem a~zagadnieniem
wyznaczania przepływu maksymalnego oraz równoważność rozwiązań obu zadań.

Na początek zdefiniujmy w~sposób formalny pojęcia użyte w~zadaniu.
Załóżmy, że dane są następujące zbiory:

\begin{itemize}
	\item \textbf{zbiór ekspertów}, oznaczony $E$,
	\item \textbf{zbiór umiejętności}, oznaczony $S$,
	\item \textbf{zbiór projektów}, oznaczony $P$.
\end{itemize}

\begin{defn}
\textbf{Funkcją umiejętności} nazywamy funkcję
$$ \ability : E \times S \to \{ 0,1 \} $$
gdzie dla eksperta $e \in E$ oraz umiejętności $s \in S$ zachodzi
$\ability(e, s) = 1$ wtedy i~tylko wtedy, gdy ekspert $e$ posiada umiejętność
$s$, zaś 0 w przeciwnym przypadku.
\end{defn}

\begin{defn}
\textbf{Zapotrzebowaniem projektu} nazywamy funkcję
$$ \need : P \times S \to \mathbb{N} $$
gdzie dla projektu $p \in P$ i~umiejętności $s \in S$ zachodzi $\need(p, s) = n$
wtedy i~tylko wtedy, gdy w~projekcie $p$ liczba potrzebnych ekspertów
w~dziedzinie umiejętności $s$ wynosi $n$.
\end{defn}

\begin{defn}
\textbf{Przyporządkowaniem eksperta} nazywamy funkcję
$$ \assign : E \to P \times S $$
gdzie dla projektu $p \in P$ i~umiejętności $s \in S$ zachodzi
$\assign(e) = (p,s)$ wtedy i~tylko wtedy, gdy:
\begin{itemize}
	\item ekspert $e$ posiada umiejętność $s$ (tj. $\ability(e,s) = 1$),
	\item ekspert $e$ został przyporządkowany do pracy w~projekcie $p$
	w~dziedzinie umiejętności $s$.
\end{itemize}
Ponadto, dla każdego projektu $p$ i umiejętności $s$ musi zachodzić
$$ \left| \assign^{-1}[p,s] \right| \leq \need(p,s) $$
gdzie $\assign^{-1}[p,s]$ jest przeciwobrazem funkcji $\assign$ dla argumentów
$p,s$.
\end{defn}

\begin{defn}
\textbf{Liczbą braków w projekcie $p$} dla danego przyporządkowania $\assign$
nazywamy liczbę
$$ \missing(p,\assign) = \sum_{s \in S} \left( \need(p,s) -
\left| \assign^{-1}[p,s] \right| \right) $$
\end{defn}

\begin{defn}
\textbf{Całkowitą liczbą braków} dla danego przyporządkowania $\assign$ nazywamy
liczbę
$$ M(\assign) = \sum_{p \in P} \missing(p, \assign) $$
\end{defn}

Widoczne jest, że $M$ jest parametrem minimalizowanym w~postawionym problemie,
zależnym od~końcowego przyporządkowania.

Na~podstawie powyższych definicji skonstruujemy teraz sieć, której użyjemy
do~wyznaczenia rozwiązań problemu.

\begin{defn}
\textbf{Siecią przydziałów} nazwiemy sieć $N = \left(G,c,s_G,t_G\right)$, gdzie:
\begin{itemize}
	\item $G = \left(V_G,E_G\right)$ jest grafem skierowanym takim, że:
	\begin{itemize}
		\item $V_G = E \cup S \cup P \cup \left\lbrace s_G,t_G \right\rbrace$,
		\item $E_G = \left\lbrace (e,s) : \ability(e,s) = 1, e \in E, s \in S
		\right\rbrace \cup \left\lbrace (s,p) : \need(s,p) > 0, s \in S, p \in P
		\right\rbrace$, tj. krawędziami połączeni są eksperci z~ich opanowanymi
		umiejętnościami, oraz projekty z~potrzebnymi do~ich realizacji
		umiejętnościami.
	\end{itemize}
	\item $c : E_G \to \mathbb{N}$ jest funkcją pojemności zdefiniowaną
	następująco:
	\begin{itemize}
		\item jeżeli $e = s_G u, u \in E$, to $c(e) = 1$,
		\item jeżeli $e = es, e \in E, s \in S$, to $c(e) = \ability(e,s) = 1$,
		\item jeżeli $e = sp, s \in S, p \in P$, to $c(e) = \need(p,s)$,
		\item jeżeli $e = pt_G, p \in P$, to
		$$c(e) = \sum_{sp \in E_G} c(up)$$
		(tj. pojemność tej krawędzi jest równa sumie pojemności krawędzi
		wchodzących do wierzchołka $p$).
	\end{itemize}
	\item $s_G,t_G$ są wyróżnionymi wierzchołkami z $V_G$ --- kolejno źródłem
	i~ujściem.
\end{itemize}
\end{defn}

\begin{lem}
Dla dowolnego przyporządkowania $\assign$ mamy
$$ M(\assign) \geq \left( \sum_{p \in P} \sum_{s \in S} \need(p,s) \right)
- |E| $$
\end{lem}

\begin{proof}
Mamy
\begin{align*}
M(\assign) &= \sum_{p \in P} \missing(p, \assign) = \\
&= \sum_{p \in P} \left( \sum_{s \in S} \need(p,s) -
\left| \assign^{-1}[p,s] \right| \right) = \\
&= \left( \sum_{p \in P} \sum_{s \in S} \need(p,s) \right) -
\left(\sum_{p \in P} \sum_{s \in S} \left| \assign^{-1}[p,s] \right|\right) = \\
&\geq \left( \sum_{p \in P} \sum_{s \in S} \need(p,s) \right) - |E|
\end{align*}
na mocy rozłączności przeciwobrazów funkcji $\assign$.
\end{proof}

\begin{thm}
Przepływ maksymalny w~sieci przydziałów wyznacza przyporządkowanie o~minimalnej
możliwej wartości parametru $M$. 
\end{thm}

\begin{proof}
Aby dowieść to twierdzenie, wykażemy kolejno, że:
\begin{enumerate}
	\item Dowolny przepływ w~sieci przydziałów wyznacza poprawne (niekoniecznie
	maksymalne) przyporządkowanie ekspertów do~zadań.
	
	Niech dany będzie pewien przepływ $f$ w~sieci przydziałów $N$.
	Przyporządkowanie ekspertów do projektów wyznaczamy w~następujący sposób:
	\begin{enumerate}
		\item Pewnego eksperta $e \in E$ przypisujemy do~umiejętności $s \in S$,
		jeżeli $f(e,s) = 1$.
		%\item Pewnemu projektowi $p \in P$ w~dziedzinie umiejętności $s \in S$
		%przypisujemy dokładnie $f(s,p)$ ekspertów.
		\item Niech dana będzie pewna umiejętność $s \in S$. Oznaczmy zbiór
		ekspertów przypisanych do~tej umiejętności w punkcie (a) jako $E_s$.

		Zbiór $E_s$ dzielimy na~rozłączne podzbiory $E_{s,p}$ takie, że
		$|E_{s,p}| = f(s,p)$.
		\item Dla każdego z uzyskanych podzbiorów $E_{s,p}$, gdzie
		$s \in S, p \in P$ określamy wartość przyporządkowania $\assign_f$ jako
		$$ (\forall e \in E_{s,p}) \assign_f(e) = (s,p) $$

	\end{enumerate}
	Zauważmy następujące fakty:
	\begin{itemize}
		\item Rozważmy wierzchołek $e \in E$ odpowiadający pewnemu ekspertowi.
		
		Z~definicji zbioru krawędzi sieci i~funkcji przepustowości,
		do~wierzchołka tego wchodzi dokładnie jedna krawędź o~pojemności 1,
		a~wychodzi z~niego co~najwyżej $|S|$ krawędzi o~pojemności 1.

		Stąd w~przepływie $f$ tylko jedna z~krawędzi wychodzących może mieć
		przepływ 1, a~więc każdy ekspert może być przyporządkowany
		do~co~najwyżej jednej umiejętności.
		\item Rozważmy dowolny wierzchołek $s \in S$ odpowiadający pewnej
		umiejętności.

		Z~własności przepływu mamy
		$$ \sum_{us \in E_G} f(us) = \sum_{sv \in E_G} f(sv) $$
		Wiedząc, że~wszystkie krawędzie wchodzące do~$s$ wychodzą ze~zbioru $E$,
		oraz że~wszystkie krawędzie wychodzące z~$s$ wchodzą do~zbioru $P$, mamy
		$$ \sum_{e \in E} f(es) = \sum_{p \in P} f(sp) $$
		Krawędzie o~niezerowym przepływie wchodzące do~$e$ reprezentują
		ekspertów przydzielonych do~danej umiejętności, zaś krawędzie
		o~niezerowym przepływie wychodzące z~$e$ reprezentują zapotrzebowanie
		projektów na~ekspertów z~umiejętnością $s$.

		Ponieważ suma przepływów krawędzi wchodzących i~wychodzących jest
		taka sama, każdego eksperta w~dziedzinie umiejętności $s$
		można przypisać do~dokładnie jednego projektu, a~więc można wykonać
		punkt (b) konstrukcji.
		\item Rozważmy dowolne dwa wierzchołki $s \in S, p \in P$ takie, że
		$sp \in E_G$. Z~definicji sieci mamy $c(s,p) = \need(s,p)$,
		a~z~konstrukcji rozwiązania wynika, że
		$f(s,p) = \left|\assign_f^{-1}[s,p]\right|$. Stąd na~mocy definicji
		przepływu mamy
		$$ \left|\assign_f^{-1}[s,p]\right| = f(s,p) \leq c(s,p) = \need(s,p)$$
		\item Rozważmy dowolny wierzchołek $p \in P$. Z~definicji funkcji
		pojemności, jeśli wszystkie krawędzie wchodzące do~$p$ będą wysycone
		przepływem (tj. $f(e) = c(e)$), to~przepływ ten można przekazać
		w~całości do ujścia krawędzią $pt_G$, bo
		$$ c(pt_G) = \sum_{up \in E_G} c(up) $$
		Stąd pojemność krawędzi $pt_G$ nie ogranicza wartości maksymalnego
		przepływu.
	\end{itemize}
	Wyznaczone przyporządkowanie $\assign_f$ spełnia więc wszystkie warunki
	prawidłowego przyporządkowania ekspertów do projektów.

	\item Skonstruowane z~przepływu maksymalnego przyporządkowanie wyznacza
	minimalną wartość parametru $M$.

	Dowód nie~wprost: Załóżmy, że~przyporządkowanie $\assign_{\max}$
	skonstruowane na~podstawie przepływu maksymalnego $f_{\max}$ w~sposób
	opisany w punkcie 1. można rozszerzyć tak, aby zmniejszyć wartość parametru
	$M$.

	Na mocy lematu mamy
	\begin{align*}
	\left( \sum_{p \in P} \sum_{s \in S} \need(p,s) \right) -
	\left(\sum_{p \in P} \sum_{s \in S} \left| \assign_{\max}^{-1} [p,s]\right|\right) 
	&= M(\assign_{\max}) > \\
	&> \left( \sum_{p \in P} \sum_{s \in S} \need(p,s) \right) - |E|
	\end{align*}
	a~więc istnieje co~najmniej jeden ekspert $e \in E$, który nie~jest
	przyporządkowany do~jednego z~projektów.
	\begin{enumerate}
		\item Jeżeli $(\forall s \in S) \ability(e,s) = 0$, to eksperta
		nie~można przyporządkować do~żadnego projektu.
		\item Załóżmy, że istnieje umiejętność $s \in S$ taka, że
		$\ability(e,s) = 1$.
		\begin{enumerate}
			\item Jeżeli $(\forall p \in P) \need(p,s) =
			\left|\assign_{\max}^{-1} [p,s]\right|$, to eksperta nie~można
			przyporządkować do~żadnego projektu.
			\item Załóżmy, że~istnieje projekt $p$ taki, że
			$\need(p,s) > \left|\assign_{\max}^{-1} [p,s]\right|$.
			Wówczas w~sieci rezydualnej dla przepływu $f_{\max}$ istnieje
			ścieżka rozszerzająca postaci $s_G e s p t_G$, wzdłuż której można
			zwiększyć przepływ o~1. Sprzeczność z~maksymalnością przepływu
			$f_{\max}$.
		\end{enumerate}
	\end{enumerate}
\end{enumerate}
\end{proof}

\end{document}
