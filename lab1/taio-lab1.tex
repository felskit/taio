\documentclass[12pt,a4paper]{article}
\usepackage[T1]{fontenc} 
\usepackage[utf8]{inputenc}
\usepackage{amsmath}
\usepackage{amsthm}
\usepackage{amssymb}
\usepackage{graphicx}
\usepackage{tabularx}
\usepackage{multirow}
\usepackage{float}
\usepackage{subcaption}
\usepackage{listings}
\usepackage{tcolorbox}
\usepackage[hidelinks]{hyperref}
\usepackage[polish]{babel}
\usepackage[useregional=numeric]{datetime2}
\usepackage[a4paper,margin=1in]{geometry}
\usepackage{tikz}
\usetikzlibrary{shapes,snakes,positioning}

\newtheorem{lem}{Lemat}
\newtheorem{thm}{Twierdzenie}
\theoremstyle{definition}
\newtheorem{defn}{Definicja}

\DeclareMathOperator{\ability}{ability}
\DeclareMathOperator{\need}{need}
\DeclareMathOperator{\assign}{assign}
\DeclareMathOperator{\missing}{missing}

\renewcommand{\lstlistingname}{Przykład}
\lstset{
	captionpos=b
}

\begin{document}

\begin{titlepage}
	\centering
	{\Large Wydział Matematyki i Nauk Informacyjnych\\Politechniki Warszawskiej \par}
	\vspace{1cm}
	\includegraphics[width=0.2\textwidth]{res/img/logo.png} \par
	\vspace{4cm}
	{\LARGE Teoria Algorytmów i Obliczeń\\Laboratorium - Etap 1 \par}
	\vspace{0.5cm}
	{\LARGE Specyfikacja wstępna \par}
	\vspace{2cm}
	{\large Adrian Bednarz,\\Bartłomiej Dach,\\Tymon Felski \par}
	\vspace{2cm}
	{\large Wersja 1.0 \par}
	\vspace{0.5cm}
	{\large \today \par}
\end{titlepage}

\noindent
Lista zmian:\\[.5\baselineskip]
\begin{tabularx}{\textwidth}{|l|l|X|l|}
	\hline
	\textbf{Data} & \textbf{Autor} & \textbf{Opis} & \textbf{Wersja} \\
	\hline
	15.10.2017 & Tymon Felski & Stworzenie szablonu dokumentu & 1.0 \\
	\hline
\end{tabularx}

\newpage
\tableofcontents
\newpage

\section{Opis problemu}
\label{sec:description}
Niniejszy rozdział poświęcony jest dokładnemu opisaniu podstawowej wersji zadanego problemu.\\

\noindent
Dane są zbiory:
\begin{itemize}
	\item $E$ - \textbf{ekspertów} realizujących projekty,
	\item $S$ - \textbf{umiejętności} posiadanych przez ekspertów,
	\item $P$ - \textbf{projektów} do zrealizowania.\\
\end{itemize}

\noindent
Każdemu ekspertowi przypisany jest wektor binarny opisujący posiadane przez niego umiejętności. Przykładowo, jeżeli ekspert posiada umiejętność $i$, to w~wektorze umiejętności odpowiadającemu temu ekspertowi na $i$-tym miejscu znajduje się znak $1$, w~przeciwnym wypadku --- $0$.\\

\begin{tcolorbox}[title=Przykład --- wektory ekspertów]
Załóżmy, że liczność zbioru umiejętności $S$ jest równa 5. Wówczas ponumerujemy umiejętności rozważane w~problemie liczbami z zakresu $[1, 5]$. Niech pewien ekspert ze~zbioru $E$ posiada umiejętności 2, 3 i~5. Wówczas wektor opisujący jego umiejętności to:
$$[0, 1, 1, 0, 1]$$
\end{tcolorbox}

\vspace{0.5em}
\noindent
Każdemu projektowi przypisany jest wektor liczbowy opisujący zapotrzebowanie na ekspertów posiadających dane umiejętności. Przykładowo, jeżeli do realizacji projektu potrzeba trzech ekspertów posiadających umiejętność $i$, to w wektorze umiejętności odpowiadającemu temu projektowi na $i$-tym miejscu znajduje się liczba $3$.\\

\begin{tcolorbox}[title=Przykład --- wektory zapotrzebowania projektów]
Utrzymując założenie o~liczności zbioru umiejętności z~poprzedniego przykładu, rozważmy pewien projekt ze~zbioru $P$. Niech jego zapotrzebowanie na~ekspertów posiadających umiejętności 1 i 4 wynosi odpowiednio 4 i 3, a~na~pozostałe --- 0. Wówczas wektor opisujący zapotrzebowanie tego projektu to:
$$[4, 0, 0, 3, 0]$$
\end{tcolorbox}

\vspace{0.5em}
\noindent
Wszystkie projekty realizowane są w tym samym oknie czasowym, tzn. prace nad każdym z nich rozpoczynają się w momencie $t_0$ i kończą w późniejszym momencie $t_k$. Oznacza to, że jeżeli dany ekspert zostanie zatrudniony do pracy nad projektem $P_1$, to nie będzie mógł brać udziału w równoległym projekcie $P_2$. Ponadto każdy ekspert podczas pracy nad projektem może wykorzystywać tylko jedną z posiadanych umiejętności i nie może jej zmienić w trakcie trwania prac.\\

\noindent
Prace nad danym projektem zostaną zakończone nawet jeżeli nie zostanie mu przydzielona wymagana liczba ekspertów posiadających potrzebne umiejętności, określona przez wektor liczbowy odpowiadający temu projektowi. Będzie on natomiast zrealizowany gorzej niż w przypadku, gdyby przypisana została odpowiednia liczba ekspertów. Może się również zdarzyć, że najbardziej optymalne okaże się takie przypisanie ekspertów, że nad pewnym projektem nie będzie pracował nikt.\\

\noindent
Jeżeli zapotrzebowanie projektu nie zostanie wypełnione w~całości, mamy do~czynienia z~brakami. Poprzez braki rozumiemy różnicę pomiędzy zapotrzebowaniem projektu na ekspertów o danych umiejętnościach a rzeczywistym przydziałem. Aby wyznaczyć braki w danym projekcie, należy odjąć wektor zapotrzebowania projektu na~ekspertów od wektora zawierającego informację o~ekspertach przydzielonych do~tego projektu i~zsumować elementy uzyskanej różnicy. Dokładna definicja tego pojęcia znajduje się w rozdziale zawierającym dowód poprawności (definicja \ref{defn:missing}).\\

\begin{tcolorbox}[title=Przykład --- obliczanie liczby braków]
Niech wektorem opisującym zapotrzebowanie pewnego projektu na ekspertów będzie:
$$[4, 0, 0, 3, 0]$$
Załóżmy, że do tego projektu zostali przypisani eksperci opisani przez wektory:
\begin{align*}
[1, 0, 1, 0, 1] & \qquad (\text{wykorzystuje umiejętność }1) \\
[1, 0, 0, 0, 1] & \qquad (\text{wykorzystuje umiejętność }1) \\
[1, 1, 0, 1, 0] & \qquad (\text{wykorzystuje umiejętność }4) \\
[0, 0, 0, 1, 1] & \qquad (\text{wykorzystuje umiejętność }4)
\end{align*}
Wówczas przydział ekspertów do tego projektu można opisać wektorem:
$$[2, 0, 0, 2, 0]$$
Braki w tym projekcie obliczymy następująco:
$$\sum([4, 0, 0, 3, 0] - [2, 0, 0, 2, 0]) = \sum([2, 0, 0, 1, 0]) = 3$$
\end{tcolorbox}

\vspace{0.5em}
\noindent
Naszym celem jest zminimalizowanie braków w obrębie wszystkich projektów (sumy wszystkich braków), czyli znalezienie optymalnego przydziału ekspertów do projektów.

\section{Problem znajdowania maksymalnego przepływu}
\label{sec:maxflow}
Okazuje się, że problem opisany w~sekcji~\ref{sec:description} można uogólnić
do~znanego problemu znajdowania maksymalnego przepływu w~sieciach. W~tej sekcji
zdefiniowane są podstawowe pojęcia potrzebne do~opisu tego problemu.

\begin{defn}
\textbf{Siecią} nazywamy czwórkę uporządkowaną $S = (G,c,s,t)$, gdzie:
\begin{itemize}
	\item $G = (V,E)$ jest grafem skierowanym,
	\item $c : E \to \mathbb{N}$ to tzw. funkcja przepustowości,
	\item $s,t \in V, s \neq t$ są dwoma wyróżnionymi wierzchołkami grafu $G$
	--- kolejno źródłem i ujściem sieci.
\end{itemize}
\end{defn}

\begin{defn}
\textbf{Przepływem} w~sieci $S$ nazywamy funkcję $f : E \to \mathbb{N}$
spełniającą następujące warunki:
\begin{enumerate}
	\item $\forall_{e \in E} \quad 0 \leqslant f(e) \leqslant c(e)$,
	\item $\displaystyle(\forall v \in V - \{ s,t \})
	\sum_{u : \; uv \in E} f(uv) = \sum_{u : \; vu \in E} f(vu)$ 
	--- tzw. prawo Kirchhoffa.
\end{enumerate}
\end{defn}

\noindent
W~ogólniejszym przypadku funkcje przepustowości i~przepływu mogą mieć wartości
nieujemne rzeczywiste, lecz założenie o~całkowitości zapewnia, że~algorytmy
wyznaczające maksymalny przepływ zawsze zakończą działanie.\\

\noindent
Prawo Kirchhoffa stanowi, że~suma wartości przepływu na~krawędziach wchodzących
do~danego wierzchołka musi być równa sumie wartości przepływu na~krawędziach
wychodzących z~tego wierzchołka.

\begin{defn}
\textbf{Wartością przepływu} $f$ w~sieci $S$ nazywamy liczbę
$$ W(f) = \sum_{u: \; su \in E} f(su) - \sum_{u: \; us \in E} f(us) $$
\end{defn}

\noindent
Powyższe definicje wystarczą, aby~zdefiniować problem maksymalnego przepływu.

\begin{defn}[Problem maksymalnego przepływu]
Dana jest sieć $S = (G,c,s,t)$. Szukamy przepływu $f$ o~maksymalnej wartości
$W(f)$, zwanego również \textbf{przepływem maksymalnym}.
\end{defn}

\noindent
Zagadnienie znajdowania maksymalnego przepływu jest rozwiązywalne przez
wiele zachłannych algorytmów opartych na metodzie Forda-Fulkersona,
polegającej na~znajdowaniu ścieżek w~tzw. sieci rezydualnej. Szczegółowy
opis jednego z~takich algorytmów znajduje się w następnej sekcji.

\section{Algorytm}
\label{sec:algorithm}
W poniższym rozdziale precyzyjnie sformułowano algorytm pozwalający na rozwiązanie dowolnego zadania w postawionym problemie. Pseudokod został podzielony na fragmenty, z którego każdy będzie rozwiązywał pewien podproblem, w celu ułatwienia opisu głównej części algorytmu.

\subsection{Konstrukcja sieci podstawowej}
Niektóre podproblemy opisane w dalszej części rozdziału będą wymagać sieci reprezentowanej przez graf skierowany, który można utworzyć na podstawie danych z zadania.\\
\begin{figure}[t!]
	\centering
	\begin{subfigure}{0.3\textwidth}
		\begin{verbatim}
3 // Liczba ekspertów
4 // Liczba umiejętności
2 // Liczba projektów
// Wektory ekspertów
[1, 0, 1, 0]
[0, 1, 0, 0]
[0, 0, 1, 1]
// Zapotrzebowanie
// projektów
[3, 0, 2, 0]
[0, 0, 1, 5]
		\end{verbatim}
		\caption{Przykładowy format pliku wejściowego dla programu}
	\end{subfigure}
	\quad
	\begin{subfigure}{0.6\textwidth}
		\begin{tikzpicture}[->,node distance=0.5cm and 1.5cm,thick]
			\tikzstyle{every node} = [fill=white]
			\tikzstyle{vertex}	   = [circle,draw=black]
			\tikzstyle{expert}     = [diamond,draw=red,text=red]
			\tikzstyle{skill}      = [rectangle,draw=teal,text=teal,minimum height=0.7cm,minimum width=0.7cm]
			\tikzstyle{project}    = [circle,draw=blue,text=blue]

			\node(s) [vertex] {$s_G$};

			\node(e2) [expert,right=of s] {$e_2$};
			\node(e1) [expert,above=of e2] {$e_1$};
			\node(e3) [expert,below=of e2] {$e_3$};

			\node(s1) [skill,above right=of e1] {$s_1$};
			\node(s2) [skill,above right=of e2] {$s_2$};
			\node(s3) [skill,below right=of e2] {$s_3$};
			\node(s4) [skill,below right=of e3] {$s_4$};

			\node(p1) [project,right=of s2] {$p_1$};
			\node(p2) [project,right=of s3] {$p_2$};

			\node(t) [vertex] [below right=of p1] {$t_G$};

			\path	(s)  edge node{1} (e1)
						edge node{1} (e2)
						edge node{1} (e3)
					(e1) edge node{1} (s1)
						edge [bend right=15] node{1} (s3)
					(e2) edge [bend left=30] node{1} (s2)
					(e3) edge node{1} (s3)
						edge node{1} (s4)
					(s1) edge node{3} (p1)
					(s3) edge node{2} (p1)
						edge node{1} (p2)
					(s4) edge node{5} (p2)
					(p1) edge node{5} (t)
					(p2) edge node{6} (t);
		\end{tikzpicture}
		\caption{Wygląd sieci skonstruowanej na~podstawie dostarczonych danych}
	\end{subfigure}
	\caption{Przykład konstrukcji sieci na~podstawie określonego zadania
	problemu}
\end{figure}
\noindent
Proponowany graf \texttt{G} będzie posiadał $|E|+|S|+|P|+2$ wierzchołków, które utworzą w nim pięć warstw. Wyróżnione zostaną dwa wierzchołki --- źródło \texttt{s} i ujście \texttt{t} sieci, z którego każdy będzie jedynym w swojej warstwie. Pozostałe trzy warstwy pomiędzy nimi będą zawierać wierzchołki reprezentujące odpowiednio ekspertów, umiejętności i projekty. Poniżej znajduje się pseudokod pozwalający na stworzenie takiego grafu.\\

\begin{tcolorbox}[title=Konstrukcja sieci podstawowej]
\begin{verbatim}
G <- graf skierowany o liczbie wierzchołków równej |E|+|S|+|P|+2
dla każdego wierzchołka e reprezentującego eksperta w G:
    dodaj krawędź (s, e) do G
    G.c[s, e] <- 1
dla każdego wierzchołka e reprezentującego eksperta w G:
    dla każdej umiejętności u posiadanej przez eksperta e:
        dodaj krawędź (e, u) do G
        G.c[e, u] <- 1
dla każdego wierzchołka p reprezentującego projekt w G:
    dla każdej umiejętności u wymaganej przez projekt p:
        dodaj krawędź (u, p) do G
        G.c[e, u] <- liczba wymaganych ekspertów z u
dla każdego wierzchołka p reprezentującego projekt w G:
    dodaj krawędź (p, t) do G
    G.c[p, t] <- suma przepustowości krawędzi wchodzących do p
\end{verbatim}
\end{tcolorbox}

\subsection{Konstrukcja sieci rezydualnej}
W celu konstruowania w sieci ścieżek rozszerzających niezbędne jest utworzenie pomocniczej sieci rezydualnej. Przepustowość krawędzi w tej sieci zależy od wartości przepływu na krawędziach oryginalnej sieci.\\

\noindent
Niech dana będzie krawędź $uv$ i przepływ $f$. Wówczas w sieci rezydualnej istnieją krawędzie:
\begin{itemize}
	\item $uv$ o przepustowości $c(uv) - f(uv)$,
	\item $vu$ o przepustowości $f(uv)$.
\end{itemize}

\noindent
Sieć rezydualna będzie aktualizowana po każdym zwiększeniu przepływu wzdłuż ścieżki powiększającej. Na początku działania algorytmu (przy zerowym przepływie) będzie ona taka sama, jak wyjściowa sieć. Wobec tego, wystarczy stworzyć sieć opisaną w poprzednim punkcie i rozszerzyć ją w sposób następujący:\\

\begin{tcolorbox}[title=Rozszerzenie konstrukcji sieci podstawowej]
\begin{verbatim}
dla każdej krawędzi (u, v) w G:
    dodaj krawędź (v, u) do G
    G.c[v, u] <- 0
\end{verbatim}
\end{tcolorbox}

\subsection{Wyszukiwanie ścieżek}
Do funkcjonowania algorytmu potrzebna jest podprocedura wyszukująca ścieżki między dwoma wierzchołkami grafu, co można dość prosto zaimplementować poprzez modyfikację przeszukiwania w głąb (ang. depth-first search, DFS). Poniżej znajduje się pseudokod żądanej podprocedury, wyszukującej ścieżki w grafie \texttt{G} od wierzchołka \texttt{u} do \texttt{w}.\\

\begin{tcolorbox}[title=Wyszukiwanie ścieżek w grafie]
\begin{verbatim}
findPath(G, u, w):
    L <- pusta lista
    dodaj u na koniec L
    jeśli depthFirstSearch(G, L, w):
        zwróć constructPath(G, L)
    zwróć pustą listę

depthFirstSearch(G, L, w):
    u <- ostatni element listy L
    jeśli u = w:
        zwróć prawdę
    dla każdej krawędzi (u, v) wychodzącej z u w G:
        jeśli v nie należy do L:
            dodaj v na koniec L
            jeśli depthFirstSearch(G, L, w):
                zwróć prawdę
            usuń v z L
    zwróć fałsz
\end{verbatim}
\end{tcolorbox}

\vspace{0.5em}
\noindent
Powyższe wyszukiwanie używa pomocniczej funkcji budującej ścieżkę na podstawie listy wierzchołków \texttt{L}, która zdefiniowana jest następująco:\\

\begin{tcolorbox}[title=Budowanie ścieżki na podstawie wierzchołków]
\begin{verbatim}
constructPath(L):
    E <- pusta lista
    u <- pierwszy element listy L
    dla każdego wierzchołka v poza pierwszym z L:
        dodaj krawędź (u, v) na koniec E
        u <- v
    zwróć E
\end{verbatim}
\end{tcolorbox}

\vspace{0.5em}
\noindent
Zaproponowany algorytm wyszukiwania jest algorytmem z powrotami. W zapisie założono, że lista L jest przekazywana przez referencję i modyfikowana we wszystkich wywołaniach rekurencyjnych.\\

\noindent
W algorytmie Edmondsa-Karpa do wyszukiwania ścieżek wykorzystywane jest przeszukiwanie wszerz. W przypadku skonstruowanego na potrzeby zadania grafu wszystkie ścieżki mają jednak tę samą długość, a ponadto DFS znajdzie ścieżkę z mniejszą złożonością pamięciową.

\subsection{Wyznaczanie przepływu maksymalnego}
Poniżej znajduje się zapis algorytmu wyznaczającego przepływ maksymalny zgodnie z zasadą opisaną przez Forda i Fulkersona. \texttt{G\_res} jest w tym przypadku siecią rezydualną wyznaczoną na podstawie sieci podstawowej skonstruowanej w pierwszym podrozdziale.\\

\begin{tcolorbox}[title=Wyznaczanie przepływu maksymalnego]
\begin{verbatim}
maxFlow(G_res):
    f <- przepływ zerowy
    powtarzaj
        L <- findPath(G_res, s, t)
        df <- inf
        dla każdej krawędzi (u, v) z L:
            jeśli G_res.c[u, v] < df:
                df <- G_res.c[u, v]
        dla każdej krawędzi (u, v) z L:
            f[u, v] += df
            G_res.c[u, v] -= df
            G_res.c[v, u] += df
    dopóki lista L jest niepusta
    zwróć f
\end{verbatim}
\end{tcolorbox}

\vspace{0.5em}
\noindent
Algorytm ten jednak nie rozwiązuje zadania wyjściowego. Optymalną wartość braków i przyporządkowanie ekspertów do zadań należy wyznaczyć na podstawie uzyskanego przepływu.

\subsection{Konstrukcja rozwiązania}
Poniższa konstrukcja przyporządkowania ekspertów do projektów korzysta z sieci podstawowej \texttt{G} (nie z sieci rezydualnej \texttt{G\_res}) i z wyznaczonego przepływu \texttt{f}.\\

\begin{tcolorbox}[title=Wzynaczanie przydziału]
\begin{verbatim}
constructAssignment(G, f):
    skills <- słownik pustych kolejek dla poszczególnych umiejętności
    L <- pusta lista
    dla każdego wierzchołka e reprezentującego eksperta w G:
        dla każdej krawędzi (e, u) wychodzącej z e w G:
            jeśli f[e, u] = 1:
                skills[u].push(e)
    dla każdego wierzchołka u reprezentującego umiejętność w G:
        dla każdej krawędzi (u, p) wychodzącej z u w G:
            dopóki f[u, p] > 0:
                e <- skills[u].pop()
                dodaj krotkę (e, u, p) na koniec L
                f[u, p] -= 1
    zwróć L
\end{verbatim}
\end{tcolorbox}

\vspace{0.5em}
\noindent
Poniższa funkcja oblicza finalną wartość braków w wyznaczonym wyżej przyporządkowaniu \texttt{L}.\\

\begin{tcolorbox}[title=Wzynaczanie braków]
\begin{verbatim}
calcLosses(G, L):
    need <- 0
    dla każdej krawędzi (p, t) wchodzącej do ujścia t w G:
        need += G.c[p, t]
    flow <- liczba elementów w liście L
    zwróć (need - flow)
\end{verbatim}
\end{tcolorbox}

\section{Dowód poprawności}
\label{sec:correctnessproof}

W~tej sekcji wykażemy związek między postawionym problemem a~zagadnieniem
wyznaczania przepływu maksymalnego oraz równoważność rozwiązań obu zadań.\\

\noindent
Na początek zdefiniujmy w~sposób formalny pojęcia użyte w~oryginalnym zadaniu.
Załóżmy, że dane są następujące zbiory:

\begin{itemize}
	\item \textbf{zbiór ekspertów}, oznaczony $E$,
	\item \textbf{zbiór umiejętności}, oznaczony $S$,
	\item \textbf{zbiór projektów}, oznaczony $P$.
\end{itemize}

\begin{defn}
\textbf{Funkcją umiejętności} nazywamy funkcję
$$ \ability : E \times S \to \{ 0,1 \} $$
gdzie dla eksperta $e \in E$ oraz umiejętności $s \in S$ zachodzi
$\ability(e, s) = 1$ wtedy i~tylko wtedy, gdy ekspert $e$ posiada umiejętność
$s$, zaś 0 w przeciwnym przypadku.
\end{defn}

\begin{defn}
\textbf{Zapotrzebowaniem projektu} nazywamy funkcję
$$ \need : P \times S \to \mathbb{N} $$
gdzie dla projektu $p \in P$ i~umiejętności $s \in S$ zachodzi $\need(p, s) = n$
wtedy i~tylko wtedy, gdy w~projekcie $p$ liczba potrzebnych ekspertów
w~dziedzinie umiejętności $s$ wynosi $n$.
\end{defn}

\begin{defn}
\textbf{Przyporządkowaniem eksperta} nazywamy funkcję
$$ \assign : E \to P \times S $$
gdzie dla projektu $p \in P$ i~umiejętności $s \in S$ zachodzi
$\assign(e) = (p,s)$ wtedy i~tylko wtedy, gdy:
\begin{itemize}
	\item ekspert $e$ posiada umiejętność $s$ (tj. $\ability(e,s) = 1$),
	\item ekspert $e$ został przyporządkowany do pracy w~projekcie $p$
	w~dziedzinie umiejętności $s$.
\end{itemize}
Ponadto, dla każdego projektu $p$ i umiejętności $s$ musi zachodzić
$$ \left| \assign^{-1}[p,s] \right| \leq \need(p,s) $$
gdzie $\assign^{-1}[p,s]$ jest przeciwobrazem funkcji $\assign$ dla argumentów
$p,s$.
\end{defn}

\begin{defn}
\label{defn:missing}
\textbf{Liczbą braków w projekcie $p$} dla danego przyporządkowania $\assign$
nazywamy liczbę
$$ \missing(p,\assign) = \sum_{s \in S} \left( \need(p,s) -
\left| \assign^{-1}[p,s] \right| \right) $$
\end{defn}

\begin{defn}
\textbf{Całkowitą liczbą braków} dla danego przyporządkowania $\assign$ nazywamy
liczbę
$$ M(\assign) = \sum_{p \in P} \missing(p, \assign) $$
\end{defn}

\noindent
Widoczne jest, że $M$ jest parametrem minimalizowanym w~postawionym problemie,
zależnym od~końcowego przyporządkowania.\\

\noindent
Na~podstawie powyższych definicji skonstruujemy teraz sieć, której użyjemy
do~wyznaczenia rozwiązań problemu.

\begin{defn}
\textbf{Siecią przydziałów} nazwiemy sieć $N = \left(G,c,s_G,t_G\right)$, gdzie:
\begin{itemize}
	\item $G = \left(V_G,E_G\right)$ jest grafem skierowanym takim, że:
	\begin{itemize}
		\item $V_G = E \cup S \cup P \cup \left\lbrace s_G,t_G \right\rbrace$,
		\item $E_G = \left\lbrace (e,s) : \ability(e,s) = 1, e \in E, s \in S
		\right\rbrace \cup \left\lbrace (s,p) : \need(s,p) > 0, s \in S, p \in P
		\right\rbrace$, tj. krawędziami połączeni są eksperci z~ich opanowanymi
		umiejętnościami, oraz projekty z~potrzebnymi do~ich realizacji
		umiejętnościami.
	\end{itemize}
	\item $c : E_G \to \mathbb{N}$ jest funkcją pojemności zdefiniowaną
	następująco:
	\begin{itemize}
		\item jeżeli $e = s_G u, u \in E$, to $c(e) = 1$,
		\item jeżeli $e = es, e \in E, s \in S$, to $c(e) = \ability(e,s) = 1$,
		\item jeżeli $e = sp, s \in S, p \in P$, to $c(e) = \need(p,s)$,
		\item jeżeli $e = pt_G, p \in P$, to
		$$c(e) = \sum_{sp \in E_G} c(up)$$
		(tj. pojemność tej krawędzi jest równa sumie pojemności krawędzi
		wchodzących do wierzchołka $p$).
	\end{itemize}
	\item $s_G,t_G$ są wyróżnionymi wierzchołkami z $V_G$ --- kolejno źródłem
	i~ujściem.
\end{itemize}
\end{defn}

\begin{lem}
Dla dowolnego przyporządkowania $\assign$ mamy
$$ M(\assign) \geq \left( \sum_{p \in P} \sum_{s \in S} \need(p,s) \right)
- |E| $$
\end{lem}

\begin{proof}
Mamy
\begin{align*}
M(\assign) &= \sum_{p \in P} \missing(p, \assign) = \\
&= \sum_{p \in P} \left( \sum_{s \in S} \need(p,s) -
\left| \assign^{-1}[p,s] \right| \right) = \\
&= \left( \sum_{p \in P} \sum_{s \in S} \need(p,s) \right) -
\left(\sum_{p \in P} \sum_{s \in S} \left| \assign^{-1}[p,s] \right|\right) = \\
&\geq \left( \sum_{p \in P} \sum_{s \in S} \need(p,s) \right) - |E|
\end{align*}
na mocy rozłączności przeciwobrazów funkcji $\assign$.
\end{proof}

\begin{thm}
Przepływ maksymalny w~sieci przydziałów wyznacza przyporządkowanie o~minimalnej
możliwej wartości parametru $M$. 
\end{thm}

\begin{proof}
Aby dowieść to twierdzenie, wykażemy kolejno, że:
\begin{enumerate}
	\item Dowolny przepływ w~sieci przydziałów wyznacza poprawne (niekoniecznie
	maksymalne) przyporządkowanie ekspertów do~zadań.
	
	Niech dany będzie pewien przepływ $f$ w~sieci przydziałów $N$.
	Przyporządkowanie ekspertów do projektów wyznaczamy w~następujący sposób:
	\begin{enumerate}
		\item Pewnego eksperta $e \in E$ przypisujemy do~umiejętności $s \in S$,
		jeżeli $f(e,s) = 1$.
		%\item Pewnemu projektowi $p \in P$ w~dziedzinie umiejętności $s \in S$
		%przypisujemy dokładnie $f(s,p)$ ekspertów.
		\item Niech dana będzie pewna umiejętność $s \in S$. Oznaczmy zbiór
		ekspertów przypisanych do~tej umiejętności w punkcie (a) jako $E_s$.

		Zbiór $E_s$ dzielimy na~rozłączne podzbiory $E_{s,p}$ takie, że
		$|E_{s,p}| = f(s,p)$.
		\item Dla każdego z uzyskanych podzbiorów $E_{s,p}$, gdzie
		$s \in S, p \in P$ określamy wartość przyporządkowania $\assign_f$ jako
		$$ (\forall e \in E_{s,p}) \assign_f(e) = (s,p) $$

	\end{enumerate}
	Zauważmy następujące fakty:
	\begin{itemize}
		\item Rozważmy wierzchołek $e \in E$ odpowiadający pewnemu ekspertowi.
		
		Z~definicji zbioru krawędzi sieci i~funkcji przepustowości,
		do~wierzchołka tego wchodzi dokładnie jedna krawędź o~pojemności 1,
		a~wychodzi z~niego co~najwyżej $|S|$ krawędzi o~pojemności 1.

		Stąd w~przepływie $f$ tylko jedna z~krawędzi wychodzących może mieć
		przepływ 1, a~więc każdy ekspert może być przyporządkowany
		do~co~najwyżej jednej umiejętności.
		\item Rozważmy dowolny wierzchołek $s \in S$ odpowiadający pewnej
		umiejętności.

		Z~własności przepływu mamy
		$$ \sum_{us \in E_G} f(us) = \sum_{sv \in E_G} f(sv) $$
		Wiedząc, że~wszystkie krawędzie wchodzące do~$s$ wychodzą ze~zbioru $E$,
		oraz że~wszystkie krawędzie wychodzące z~$s$ wchodzą do~zbioru $P$, mamy
		$$ \sum_{e \in E} f(es) = \sum_{p \in P} f(sp) $$
		Krawędzie o~niezerowym przepływie wchodzące do~$e$ reprezentują
		ekspertów przydzielonych do~danej umiejętności, zaś krawędzie
		o~niezerowym przepływie wychodzące z~$e$ reprezentują zapotrzebowanie
		projektów na~ekspertów z~umiejętnością $s$.

		Ponieważ suma przepływów krawędzi wchodzących i~wychodzących jest
		taka sama, każdego eksperta w~dziedzinie umiejętności $s$
		można przypisać do~dokładnie jednego projektu, a~więc można wykonać
		punkt (b) konstrukcji.
		\item Rozważmy dowolne dwa wierzchołki $s \in S, p \in P$ takie, że
		$sp \in E_G$. Z~definicji sieci mamy $c(s,p) = \need(s,p)$,
		a~z~konstrukcji rozwiązania wynika, że
		$f(s,p) = \left|\assign_f^{-1}[s,p]\right|$. Stąd na~mocy definicji
		przepływu mamy
		$$ \left|\assign_f^{-1}[s,p]\right| = f(s,p) \leq c(s,p) = \need(s,p)$$
		\item Rozważmy dowolny wierzchołek $p \in P$. Z~definicji funkcji
		pojemności, jeśli wszystkie krawędzie wchodzące do~$p$ będą wysycone
		przepływem (tj. $f(e) = c(e)$), to~przepływ ten można przekazać
		w~całości do ujścia krawędzią $pt_G$, bo
		$$ c(pt_G) = \sum_{up \in E_G} c(up) $$
		Stąd pojemność krawędzi $pt_G$ nie ogranicza wartości maksymalnego
		przepływu.
	\end{itemize}
	Wyznaczone przyporządkowanie $\assign_f$ spełnia więc wszystkie warunki
	prawidłowego przyporządkowania ekspertów do projektów.

	\item Skonstruowane z~przepływu maksymalnego przyporządkowanie wyznacza
	minimalną wartość parametru $M$.

	Dowód nie~wprost: Załóżmy, że~przyporządkowanie $\assign_{\max}$
	skonstruowane na~podstawie przepływu maksymalnego $f_{\max}$ w~sposób
	opisany w punkcie 1. można rozszerzyć tak, aby zmniejszyć wartość parametru
	$M$.

	Na mocy lematu mamy
	\begin{align*}
	\left( \sum_{p \in P} \sum_{s \in S} \need(p,s) \right) -
	\left(\sum_{p \in P} \sum_{s \in S} \left| \assign_{\max}^{-1} [p,s]\right|\right) 
	&= M(\assign_{\max}) > \\
	&> \left( \sum_{p \in P} \sum_{s \in S} \need(p,s) \right) - |E|
	\end{align*}
	a~więc istnieje co~najmniej jeden ekspert $e \in E$, który nie~jest
	przyporządkowany do~jednego z~projektów.
	\begin{enumerate}
		\item Jeżeli $(\forall s \in S) \ability(e,s) = 0$, to eksperta
		nie~można przyporządkować do~żadnego projektu.
		\item Załóżmy, że istnieje umiejętność $s \in S$ taka, że
		$\ability(e,s) = 1$.
		\begin{enumerate}
			\item Jeżeli $(\forall p \in P) \need(p,s) =
			\left|\assign_{\max}^{-1} [p,s]\right|$, to eksperta nie~można
			przyporządkować do~żadnego projektu.
			\item Załóżmy, że~istnieje projekt $p$ taki, że
			$\need(p,s) > \left|\assign_{\max}^{-1} [p,s]\right|$.
			Wówczas w~sieci rezydualnej dla przepływu $f_{\max}$ istnieje
			ścieżka rozszerzająca postaci $s_G e s p t_G$, wzdłuż której można
			zwiększyć przepływ o~1. Sprzeczność z~maksymalnością przepływu
			$f_{\max}$.
		\end{enumerate}
	\end{enumerate}
\end{enumerate}
\end{proof}

\end{document}
